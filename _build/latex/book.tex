%% Generated by Sphinx.
\def\sphinxdocclass{jupyterBook}
\documentclass[letterpaper,10pt,english]{jupyterBook}
\ifdefined\pdfpxdimen
   \let\sphinxpxdimen\pdfpxdimen\else\newdimen\sphinxpxdimen
\fi \sphinxpxdimen=.75bp\relax
\ifdefined\pdfimageresolution
    \pdfimageresolution= \numexpr \dimexpr1in\relax/\sphinxpxdimen\relax
\fi
%% let collapsible pdf bookmarks panel have high depth per default
\PassOptionsToPackage{bookmarksdepth=5}{hyperref}
%% turn off hyperref patch of \index as sphinx.xdy xindy module takes care of
%% suitable \hyperpage mark-up, working around hyperref-xindy incompatibility
\PassOptionsToPackage{hyperindex=false}{hyperref}
%% memoir class requires extra handling
\makeatletter\@ifclassloaded{memoir}
{\ifdefined\memhyperindexfalse\memhyperindexfalse\fi}{}\makeatother

\PassOptionsToPackage{warn}{textcomp}

\catcode`^^^^00a0\active\protected\def^^^^00a0{\leavevmode\nobreak\ }
\usepackage{cmap}
\usepackage{fontspec}
\defaultfontfeatures[\rmfamily,\sffamily,\ttfamily]{}
\usepackage{amsmath,amssymb,amstext}
\usepackage{polyglossia}
\setmainlanguage{english}



\setmainfont{FreeSerif}[
  Extension      = .otf,
  UprightFont    = *,
  ItalicFont     = *Italic,
  BoldFont       = *Bold,
  BoldItalicFont = *BoldItalic
]
\setsansfont{FreeSans}[
  Extension      = .otf,
  UprightFont    = *,
  ItalicFont     = *Oblique,
  BoldFont       = *Bold,
  BoldItalicFont = *BoldOblique,
]
\setmonofont{FreeMono}[
  Extension      = .otf,
  UprightFont    = *,
  ItalicFont     = *Oblique,
  BoldFont       = *Bold,
  BoldItalicFont = *BoldOblique,
]



\usepackage[Bjarne]{fncychap}
\usepackage[,numfigreset=1,mathnumfig]{sphinx}

\fvset{fontsize=\small}
\usepackage{geometry}


% Include hyperref last.
\usepackage{hyperref}
% Fix anchor placement for figures with captions.
\usepackage{hypcap}% it must be loaded after hyperref.
% Set up styles of URL: it should be placed after hyperref.
\urlstyle{same}


\usepackage{sphinxmessages}



        % Start of preamble defined in sphinx-jupyterbook-latex %
         \usepackage[Latin,Greek]{ucharclasses}
        \usepackage{unicode-math}
        % fixing title of the toc
        \addto\captionsenglish{\renewcommand{\contentsname}{Contents}}
        \hypersetup{
            pdfencoding=auto,
            psdextra
        }
        % End of preamble defined in sphinx-jupyterbook-latex %
        

\title{My sample book}
\date{Nov 07, 2024}
\release{}
\author{basics}
\newcommand{\sphinxlogo}{\vbox{}}
\renewcommand{\releasename}{}
\makeindex
\begin{document}

\pagestyle{empty}
\sphinxmaketitle
\pagestyle{plain}
\sphinxtableofcontents
\pagestyle{normal}
\phantomsection\label{\detokenize{intro::doc}}


\sphinxAtStartPar
Questo libro fa parte del materiale pensato per \sphinxhref{https://basics2022.github.io/bbooks-hs}{le scuole superiori}. E’ disponibile la \sphinxhref{https://www.github.com/Basics2022/bbooks-chemistry-hs/blob/master/\_build/latex/book.pdf}{versione in .pdf} scaricabile.

\sphinxAtStartPar
Valutare se fare uno o più bbook di “scienze applicate”, per astronomia, scienza della terra, biologia

\sphinxAtStartPar
\sphinxstylestrong{Programma sintetico.}



\sphinxAtStartPar
\sphinxstylestrong{1}
\begin{itemize}
\item {} 
\sphinxAtStartPar
\sphinxstylestrong{Cenni di astronomia e scienze della terra}…

\item {} 
\sphinxAtStartPar
Introduzione al metodo scientifico e alla chimica

\item {} 
\sphinxAtStartPar
Costituzione della materia

\item {} 
\sphinxAtStartPar
Modelli e leggi della chimica: teoria atomica, cons.massa, prop def e multiple,…

\item {} 
\sphinxAtStartPar
Reazioni chimiche

\item {} 
\sphinxAtStartPar
Stati della materia: esperimenti e leggi sui gas

\end{itemize}

\sphinxAtStartPar
\sphinxstylestrong{2}
\begin{itemize}
\item {} 
\sphinxAtStartPar
Introduzione alla chimica

\item {} 
\sphinxAtStartPar
Materia: stati, transizioni; modello atomico

\item {} 
\sphinxAtStartPar
Sostanze e miscugli

\item {} 
\sphinxAtStartPar
Reazioni: Lavoisier, proporzioni definite e multiple

\item {} 
\sphinxAtStartPar
Modelli atomici

\item {} 
\sphinxAtStartPar
Tavola periodica

\item {} 
\sphinxAtStartPar
Legami chimici

\end{itemize}

\sphinxAtStartPar
\sphinxstylestrong{3}
\begin{itemize}
\item {} 
\sphinxAtStartPar
Struttura atomica

\item {} 
\sphinxAtStartPar
Tavola periodica e reattività elementi

\item {} 
\sphinxAtStartPar
Legami chimici

\item {} 
\sphinxAtStartPar
Mole e concentrazione

\item {} 
\sphinxAtStartPar
Proprietà colligative

\item {} 
\sphinxAtStartPar
Velocità di una reazione: reazioni spontane e no; catalizzatori, fattori che influenzano reazione,…

\item {} 
\sphinxAtStartPar
\sphinxstylestrong{Biologia}…

\end{itemize}

\sphinxAtStartPar
\sphinxstylestrong{4}
\begin{itemize}
\item {} 
\sphinxAtStartPar
Equilibrio chimico

\item {} 
\sphinxAtStartPar
Acidi, basi e pH

\item {} 
\sphinxAtStartPar
Termodinamica delle reazioni

\item {} 
\sphinxAtStartPar
Elettrochimica

\item {} 
\sphinxAtStartPar
\sphinxstylestrong{Biologia}…

\item {} 
\sphinxAtStartPar
\sphinxstylestrong{Scienze della Terra}…

\end{itemize}

\sphinxAtStartPar
\sphinxstylestrong{5}
\begin{itemize}
\item {} 
\sphinxAtStartPar
\sphinxstylestrong{Chimica organica}: alcani, alcheni, alchini; composti aromatici; gruppi funzionali

\item {} 
\sphinxAtStartPar
\sphinxstylestrong{Biochimica e biotecnologie}

\item {} 
\sphinxAtStartPar
\sphinxstylestrong{Scienze della Terra}

\end{itemize}

\sphinxstepscope


\chapter{Argomenti}
\label{\detokenize{ch/units:argomenti}}\label{\detokenize{ch/units:book-chemistry-hs-topics}}\label{\detokenize{ch/units::doc}}

\section{Introduzione alla scienza}
\label{\detokenize{ch/units:introduzione-alla-scienza}}\begin{itemize}
\item {} 
\sphinxAtStartPar
Metodo scientifico, grandezze fisiche, unità di misura, statistica ed errori,…
\sphinxstylestrong{todo} aggiungere collegamente all’introduzione al metodo scientifico del bbook\sphinxhyphen{}physics\sphinxhyphen{}hs

\end{itemize}


\section{Introduzione all teoria atomica}
\label{\detokenize{ch/units:introduzione-all-teoria-atomica}}\begin{itemize}
\item {} 
\sphinxAtStartPar
Breve storia:
\begin{itemize}
\item {} 
\sphinxAtStartPar
17xx,Lavoisier: conservazione massa

\item {} 
\sphinxAtStartPar
1799, Proust: proporzioni definite: nelle reazioni che producono un determinato prodotto, i reagenti si combinano in proporzioni costanti e ben definiti
\begin{itemize}
\item {} 
\sphinxAtStartPar
eccezione: Berthollet e i composti non\sphinxhyphen{}stechiometrici: solidi possono avere composizione non defiinita a causa dei difetti cristallini

\end{itemize}

\item {} 
\sphinxAtStartPar
1804, Dalton: proporzioni multiple

\item {} 
\sphinxAtStartPar
1805\sphinxhyphen{}1815legge dei volumi di Gay\sphinxhyphen{}Lussac, Avogadro e il concetto di molecola. in \sphinxstylestrong{Stati della materia}

\end{itemize}

\end{itemize}

\sphinxAtStartPar
\sphinxstylestrong{todo} fare riferimento all’introduzione storica nel bbook\sphinxhyphen{}physics\sphinxhyphen{}hs:thermodynamics, e al video di Bressanini “vedere l’atomo”


\section{Stati della materia}
\label{\detokenize{ch/units:stati-della-materia}}\begin{itemize}
\item {} 
\sphinxAtStartPar
Gas: leggi di Boyle, Charles, Gay\sphinxhyphen{}Lussac, Avogadro, gas perfetti; teoria cinetica; gas reali

\item {} 
\sphinxAtStartPar
Liquidi

\item {} 
\sphinxAtStartPar
Solidi

\end{itemize}


\section{“Ordinamento” degli elementi chimici}
\label{\detokenize{ch/units:ordinamento-degli-elementi-chimici}}\begin{itemize}
\item {} 
\sphinxAtStartPar
1869\sphinxhyphen{}71,
\sphinxstylestrong{Tavola periodica degli elementi di Mendeleev} \sphinxstylestrong{todo} \sphinxstyleemphasis{riferimento a F ckin genius?}, e caratteristiche atomi:
\begin{itemize}
\item {} 
\sphinxAtStartPar
classificazione iniziata nella prima metà del XIX secolo

\item {} 
\sphinxAtStartPar
Proprietà:
\begin{itemize}
\item {} 
\sphinxAtStartPar
peso atomico std

\item {} 
\sphinxAtStartPar
numero atomico

\item {} 
\sphinxAtStartPar
prima energia di ionizzazione

\item {} 
\sphinxAtStartPar
elettronegatività: tendenza di un atomo di attrarre verso sè \(e^-\) condivisi

\item {} 
\sphinxAtStartPar
stati di ossidazione \sphinxstylestrong{comuni}

\item {} 
\sphinxAtStartPar
config elettronica

\end{itemize}

\item {} 
\sphinxAtStartPar
Raggruppamento:
\begin{itemize}
\item {} 
\sphinxAtStartPar
gruppi, colonne: elementi con stessa config elettronica esterna. Poiché le proprietà chimiche dipendono principalmente dalla config. elettronica, gli elementi nello stesso gruppo hanno caratteristiche chimiche simili

\item {} 
\sphinxAtStartPar
periodi, righe: elementi con lo stesso livello energetico; alcuni insiemi di elementi appartenenti allo stesso periodo mostrano proprietà simili, come il blocco \(f\) dei lantanidi e gli attinidi; nello stesso periodo ci sono variazioni monotone di:
\begin{itemize}
\item {} 
\sphinxAtStartPar
raggio atomico: diminuisce, all’aumentare del numero di \(e^-\) e di \(p\), poiché aumenta l’attrazione elettrica

\item {} 
\sphinxAtStartPar
energia di ionizzazione: aumenta, al diminuire del raggio e all’aumentare dell’attrazione elettrica, poiché diventa più difficile strappare un \(e^-\) all’atomo: serve una maggiore energia per allontanare un \(e^-\)

\item {} 
\sphinxAtStartPar
elettronegatività: aumenta, all’aumentare dell’attrazione esercitata dagli \(e^-\) sul nucleo

\item {} 
\sphinxAtStartPar
affinità elettronica: … \sphinxstylestrong{todo} come?

\end{itemize}

\end{itemize}

\item {} 
\sphinxAtStartPar
Configurazione elettronica esterna

\end{itemize}

\end{itemize}


\section{Modelli atomici}
\label{\detokenize{ch/units:modelli-atomici}}
\sphinxAtStartPar
\sphinxstylestrong{Esperimenti.}
\begin{itemize}
\item {} 
\sphinxAtStartPar
Esperimenti e modelli, dall’atomo di Thompson all’atomo di Bohr ai primordi della meccanica quantistica; meccanica quantistica come teoria meccanica per la descrizione dell’atomo (1924: De Broglie; 1925: Heisenberg, Born, Jordan; 1926: Schrodinger;…)

\item {} 
\sphinxAtStartPar
Rivisitazione tavola periodica, alla luce dei modelli atomici

\end{itemize}


\section{Legami chimici}
\label{\detokenize{ch/units:legami-chimici}}\begin{itemize}
\item {} 
\sphinxAtStartPar
\sphinxstylestrong{Regola ottetto} e notazione di \sphinxstylestrong{Lewis}
\begin{itemize}
\item {} 
\sphinxAtStartPar
regola per spiegare in maniera approssimata la formazione di legami chimici;

\item {} 
\sphinxAtStartPar
tendenza a completare il livello elettronico esterno (“guscio di valenza”), per raggiungere una configurazione particolarmente stabile dal punto di vista energetico, e impedire la formazione di ulteriori legami;

\item {} 
\sphinxAtStartPar
gli elementi dei primi gruppi della tavola periodica tendono a perdere \(e^-\)

\item {} 
\sphinxAtStartPar
gli elementi dei gruppi VI, VII tende ad aquisire \(e^-\), liberando energia, chiamata \sphinxstylestrong{affinità elettronica}

\item {} 
\sphinxAtStartPar
i gas nobili, l’He e gli elementi del gruppo VIII, hanno il guscio di valenza completo e tendono a non reagire

\item {} 
\sphinxAtStartPar
idrogeno, litio e berillio, elementi “vicini” a He, raggiungono la configurazione completa con 2 \(e^-\), detta \sphinxstylestrong{duetto}

\item {} 
\sphinxAtStartPar
metalli di transizione e a partire dal terzo periodo, gli elementi hanno guscio di valenza che può ospitare un numero maggiore di \(e^-\), \sphinxstyleemphasis{“ottetto espanso”}

\item {} 
\sphinxAtStartPar
\sphinxstylestrong{spiegazione in QM}: l’energia degli orbitali è determinata “quasi esclusivamente”, a parte la \sphinxstyleemphasis{struttura fine}, dal numero quantico principale. Il numero di stati con lo stesso numero quantico principale \(n\) è \(2 \, n^2\) (2 dal principio di Pauli): quindi \(2, 8, 18,...\). La differenza di energia di livelli con \(n\) diverso è elevata,…\sphinxstylestrong{todo} \sphinxstyleemphasis{salto di energia per attrarre ulteriori \(e^-\) quando il guscio di valenza è pieno}

\item {} 
\sphinxAtStartPar
\sphinxstylestrong{Eccezioni}: \(s\), \(d\),…

\end{itemize}

\item {} 
\sphinxAtStartPar
Tipi di legami intramolecolari: legame ionico, covalente puro e polare

\item {} 
\sphinxAtStartPar
Interazioni tra molecole: legame idrogeno, …\sphinxstylestrong{todo} \sphinxstyleemphasis{altri legami?}

\item {} 
\sphinxAtStartPar
Teorie per la descrizione del legame e geometria molecolare,
\begin{itemize}
\item {} 
\sphinxAtStartPar
VSEPR

\item {} 
\sphinxAtStartPar
teoria del legame di valenza

\item {} 
\sphinxAtStartPar
…

\end{itemize}

\end{itemize}

\sphinxAtStartPar
\sphinxstylestrong{Miscele.}
\begin{itemize}
\item {} 
\sphinxAtStartPar
omogenee ed eterogenee; tecniche di separazione: filtro, centrifuga, decanter, cromatografo, distillazione

\end{itemize}

\sphinxAtStartPar
\sphinxstylestrong{Reazioni e trasformazioni della materia.}
\begin{itemize}
\item {} 
\sphinxAtStartPar
Formalismo

\item {} 
\sphinxAtStartPar
Legge di Lavoisier, proporzioni def e multiple

\item {} 
\sphinxAtStartPar
Calcolo reazioni chimiche: reagente limitante e in eccesso,

\item {} 
\sphinxAtStartPar
Soluzioni, concentrazioni: legge di Raoult, legge di Henry,…

\end{itemize}

\sphinxAtStartPar
\sphinxstylestrong{Reazioni e termodinamica.}
\begin{itemize}
\item {} 
\sphinxAtStartPar
introduzione alla termodinamica Collegamento al bbook di fisica

\item {} 
\sphinxAtStartPar
Gibbs\sphinxhyphen{}Duhem

\item {} 
\sphinxAtStartPar
legge di Hess ed entalpia di reazione

\item {} 
\sphinxAtStartPar
energia libera di Gibbs e spontaneità di una reazione

\end{itemize}

\sphinxAtStartPar
\sphinxstylestrong{Cinetica chimica}
\begin{itemize}
\item {} 
\sphinxAtStartPar
velocità reazione

\item {} 
\sphinxAtStartPar
reazioni spontanee e non; reazioni reversibili

\item {} 
\sphinxAtStartPar
catalizzatori e altri fattori che influenzano la velocità di reazione

\end{itemize}

\sphinxAtStartPar
\sphinxstylestrong{Equilibrio chimico.}
\begin{itemize}
\item {} 
\sphinxAtStartPar
Equilibrio dinamico, legge di azione di massa;

\item {} 
\sphinxAtStartPar
Fattori che influenzano l’equilibrio chimico, il principio di Le Chatelier

\item {} 
\sphinxAtStartPar
concentrazioni agli equilibri, solubilità (sale in acqua), formazione precipitato, effetto dello ione comune sulla solubilità del sale

\end{itemize}

\sphinxAtStartPar
\sphinxstylestrong{Acidi, basi e pH}
\begin{itemize}
\item {} 
\sphinxAtStartPar
def; scala pH; acidi e basi deboli e forti; soluzione tampone; pH soluzione; titolazione acido\sphinxhyphen{}base; indicatore nelle titolazioni;

\item {} 
\sphinxAtStartPar
idrolisi

\end{itemize}

\sphinxAtStartPar
\sphinxstylestrong{Elettrochimica}
\begin{itemize}
\item {} 
\sphinxAtStartPar
Numero di ossidazione; ossidante, riducente; ossidazione, riduzione; redox (forma molecolare e ionica); potere ox e red

\item {} 
\sphinxAtStartPar
energia e spontaneità; coppie redox; elettrodi; celle galvaniche; pila di Daniell; trasformazionee energia chimica/elettrica/”termica”; potenziali di cella

\item {} 
\sphinxAtStartPar
equazione di Nerst; celle elettrolitiche; leggi di Faraday

\end{itemize}

\sphinxAtStartPar
\sphinxstylestrong{Chimica organica.}
\begin{itemize}
\item {} 
\sphinxAtStartPar
Alcani, alcheni e alchini

\item {} 
\sphinxAtStartPar
Composti aromatici

\item {} 
\sphinxAtStartPar
Gruppi funzionali

\end{itemize}







\renewcommand{\indexname}{Index}
\printindex
\end{document}